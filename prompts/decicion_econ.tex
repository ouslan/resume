\documentclass[10pt, oneside]{article}
\usepackage{amsmath, amsthm, amssymb, calrsfs, wasysym, verbatim, bbm, color, graphics, geometry}

\geometry{tmargin=.75in, bmargin=.75in, lmargin=.75in, rmargin=.75in}

\newcommand{\R}{\mathbb{R}}
\newcommand{\C}{\mathbb{C}}
\newcommand{\Z}{\mathbb{Z}}
\newcommand{\N}{\mathbb{N}}
\newcommand{\Q}{\mathbb{Q}}
\newcommand{\Cdot}{\boldsymbol{\cdot}}

\newtheorem{thm}{Theorem}
\newtheorem{defn}{Definition}
\newtheorem{conv}{Convention}
\newtheorem{rem}{Remark}
\newtheorem{lem}{Lemma}
\newtheorem{cor}{Corollary}

\title{Decision Economics Prompts}
\author{Alejandro Ouslan}

\begin{document}

\maketitle

\vspace{.25in}

\section{Communication and Writing Skills}


As Lead Developer at the Puerto Rico Planning Board, I coordinated a team of research assistants (RAs) from software engineering and computer science backgrounds.
I served as a communication bridge between the Ph.D economists and the engineers, creating reproducible data processing pipelines, dividing responsibilities,
and ensuring efficient workflow execution.I also designed and implemented systems to automate the acquisition and processing of data from external agencies.
These systems generated visualizations and made datasets available in multiple formats to support internal and public use.

\section{Technical Skills}


One major responsibility during the GDP calculation for Puerto Rico was processing historical trade data, including import and export transactions
reported monthly. To manage this, I engineered a system to download, store, and convert the raw transactional data into a structured SQL database
optimized for analysis.

Additionally, I developed custom pipelines to efficiently query, clean, and transform data into panel structures suitable for econometric modeling.
I’ve worked extensively with optimization techniques. The most used technique used in the projects was the Hodrick-Prescott filter, whcih we used to
to remove short-term fluctuations from varios indicators.
\end{document}
